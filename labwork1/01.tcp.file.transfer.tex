\documentclass{article}
\usepackage[utf8]{inputenc}
\usepackage{minted}
\usepackage{graphicx}
\usepackage{indentfirst}
\usemintedstyle{colorful}

\title{Practical Work 1: TCP File transfer}
\author{Vu Dinh Anh}
\date{February 2021}

\begin{document}

\maketitle

I failed to modify provided chat system but I succeed in transferring file with TCP.

\section*{File transfer system}

Files: \mintinline{text}{ft_client.txt}, \mintinline{text}{ft_server.txt}

\subsection*{Protocol}

\begin{figure}[h]
    \centering
    \includegraphics[scale=0.3]{DS_labwork1.png}
    \caption{Protocol}
    \label{fig:protocol}
\end{figure}

\subsection*{System}

\begin{minted}{text}
+--------+
| Client |
+---+----+
    |
    | TCP/IP
    |
    v
+---+----+
| Server |
+--------+

\end{minted}

The server only listen to one client. The client send data as chunks to the server. After receiving a chunk, the server writes to a file. After finishing writing the file, the server closes itself.

\subsection*{Implementation}
In \mintinline{text}{ft_client.c}, the function to send file is implemented as followed
\begin{minted}{c}
void send_file(FILE *fp, int sockfd)
{
    char data[SIZE] = {0};

    while(fgets(data, SIZE, fp) != NULL) 
    {
        if (send(sockfd, data, sizeof(data), 0) == -1) 
        {
            perror("Error in sending file.");
            exit(1);
        }
        bzero(data, SIZE);
    }
}
\end{minted}

In \mintinline{text}{ft_server.c}, the function to write file is implemented as followed
\begin{minted}{c}
void write_file(int sockfd)
{
    int n;
    FILE *fp;
    char *filename = "recv_guid_lists.txt";
    char buffer[SIZE];

    fp = fopen(filename, "w");
    while (1) 
    {
        n = recv(sockfd, buffer, SIZE, 0);
        if (n <= 0)
        {
            break;
            return; // End the function before writing null to the file
        }
        fprintf(fp, "%s", buffer);
        bzero(buffer, SIZE);
    }
}
\end{minted}

\end{document}
