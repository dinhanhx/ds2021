\section{Definitions}
\subsection{Remote shell}

The Remote shell (rsh) \cite{remoteshhell} is a command line computer program that can execute shell commands on another computer across a computer network called the hostname machine as another user.

To do so, the remote shell must connect to a service called Remote shell daemon (rshd) on the hostname machine. Typically, this daemon uses the Transmission control protocol (TCP) port number 514.

\subsection{Message passing interface}

Message passing interface (MPI) \cite{mpi} is a portable message-passing standard designed for academia and industry to function on a myriad of parallel computing architectures. It is meant to provide essential virtual topology, synchronization and communication functionality between a set of processes that has been mapped to nodes, servers or computer instances.

MPI consists of:
\begin{itemize}
    \item a header file mpi.h
    \item a core library of routines and functions
    \item a runtime system
\end{itemize}

