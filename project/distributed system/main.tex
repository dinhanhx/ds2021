\documentclass{report}
\usepackage{import}
\usepackage{dependencies}
\usepackage[margin=2.5cm]{geometry}
\usepackage[backend=biber,
style=ieee,
citestyle=ieee]{biblatex}

\usepackage{minted}
\usepackage{graphicx}
\usepackage{indentfirst}
\usemintedstyle{colorful}


\addbibresource{references.bib}
\graphicspath{ {./images/} }

\begin{document}

\begin{titlepage}
    \title{
        \Large\MakeUppercase{distributed system final report} \\
        \Huge{\textbf{Remote shell with MPI}} \\
        %% \LARGE{} \\
        {\includegraphics[width=0.4\textwidth]{images/logo.png}}
    }
    \author{Group 5 - ICT \\
        University of Science and Technology of Hanoi\\
        \\
        
        \selectlanguage{vietnamese}
    \begin{center}
                \begin{tabular}{l c}
                    Vũ Đinh Anh & BI9 - 037 \\ [1ex]
                    Nguyễn Lan Hương & BI9 - 114 \\ [1ex]
                    Nguyễn Hồng Quang & BI9 - 194 \\
                \end{tabular}
            \end{center}
            
            }
    \date{March 2021}
\end{titlepage}

\maketitle

% \selectlanguage{english}

\newpage
\tableofcontents
\newpage

\chapter{Introduction}
\import{sections/chapter1/}{definitions}
\import{sections/chapter1/}{objective}

\chapter{Method}

\import{sections/chapter2/}{environment}
\import{sections/chapter2/}{security}
\import{sections/chapter2/}{mpi}

\section{Conclusion}

Although the program that we built is extremely simple, it is still able to illustrate the basic essence of a remote shell MPI between a client and a server. Another advantage of it is that it also incorporated a simple secure system, which is the RSA scheme.

However, we still have not developed to evaluate the performance of the system or taken into consideration problems such as data loss during transmission or incapacity to deal with a large number of requests and so on.

\printbibliography

\end{document}
